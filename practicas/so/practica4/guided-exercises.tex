\section{Ejercicios prácticos}

\subsection{Requisitos}
Para poder realizar los ejercicios prácticos debera contar con las siguientes herramientas:
\begin{itemize}
 \item El \textit{Java SE Development Kit}. Descarguelo desde la última versión desde \href{http://www.oracle.com/technetwork/java/javase/downloads}{esté}\footnote{http://www.oracle.com/technetwork/java/javase/downloads} link. Elija la arquitectura correspondiente y el formato que desee:
 \begin{itemize}
      \item Instálelo. Puede guiarse con \href{https://www.java.com/en/download/help/download\_options.xml}{esté}\footnote{https://www.java.com/en/download/help/download\_options.xml} sitio.
 \end{itemize}

 \item El \textit{SDK} de Android. Descargue la última versión desde \href{https://developer.android.com/studio/\#downloads}{esta}\footnote{https://developer.android.com/studio/\#downloads} página:
 \begin{itemize}
      \item Si decide descargar \textit{Android Studio}, ya se le instalaran todas las herramientas necesarias de manera automática. Para realizar esta práctica, solo tendrá que asociar los binarios correspondientes a la variable de entorno \textit{PATH}.
      \item De lo contrario descargue solo las herramientas de consola. Con lo cual, tendrá que desempaquetar y descomprimir los archivos correspondientes.
      
      \item Acceda al \textit{Android SDK Manager}\footnote{Para ejecutar el comando directamente agregue la herramienta ``\textit{android}'' a la variable \textit{\$PATH}.} \footnote{Si instalo Android Studio, podrá instalar las dependencias desde una ventana gráfica.}:
      \begin{lstlisting}
# sdkmanager
      \end{lstlisting}
     
      \item Descargue e instale/actualice los siguientes paquetes:
      \begin{itemize}
  \item \textit{Android SDK Tools}.
  \item \textit{Android SDK Platform-tools}.
  \item \textit{Android SDK Build-tools}.
  \item Android \ANDROIDVERSION(\ANDROIDVERSIONAPI) $\rightarrow$ \textit{SDK Platform}.
  \item Android Emulator.
      \end{itemize}

      \item Para facilitar el desarrollo de los ejercicios, agregue las siguientes herramientas a la varaible de entorno \textit{PATH}:
      \begin{itemize}
  \item \textit{<android-sdk-directory>/platform-tools/adb}.
  \item \textit{<android-sdk-directory>/platform-tools/fastboot}.
  \item \textit{<android-sdk-directory>/platform-tools/sqlite3}.
  \item \textit{<android-sdk-directory>/tools/android}.
  \item \textit{<android-sdk-directory>/tools/emulator-x86}\footnote{Puede encontrarse con el nombre ``emulator''.}.
      \end{itemize}
 \end{itemize}
 
 \item Si no instaló Android Studio, va a necesitar instalar \textit{Gradle} de manera manual. Descarguela desde \href{https://gradle.org/install/}{esté}\footnote{https://gradle.org/install/} link. Tenga en cuenta \href{https://developer.android.com/studio/releases/gradle-plugin}{esta tabla}\footnote{https://developer.android.com/studio/releases/gradle-plugin}.

 \item \textit{Linux image tools - mkbootimg} sirve para crear una imagen \textit{boot.img}. Descargue las mismas desde \href{https://github.com/unlp-so/contenidos/blob/master/practicas/so/practica4/rooting\_tools/mkbootimg}{esté}\footnote{https://github.com/unlp-so/contenidos/blob/master/practicas/so/practica4/rooting\_tools/mkbootimg} link:
 \begin{itemize}
      \item Desempaquete y descomprima el archivo.

      \item Para facilitar el desarrollo de los ejercicios, agregue a algún directorio de la varaible de entorno \textit{PATH}, el \textit{path absoluto} de la herramienta:
      \begin{itemize}
  \item \textit{<linux-image-tools-directory>/mkbootimg}.
      \end{itemize}
 \end{itemize}
 
 \item Herramienta para separar una imagen \textit{boot.img}. Descargue la última versión desde \href{https://github.com/unlp-so/contenidos/blob/master/practicas/so/practica4/rooting\_tools/unmkbootimg}{esté}\footnote{https://github.com/unlp-so/contenidos/blob/master/practicas/so/practica4/rooting\_tools/unmkbootimg} sitio:
 \begin{itemize}
      \item Descomprima el archivo.

      \item Para facilitar el desarrollo de los ejercicios, agregue a algún directorio de la varaible de entorno \textit{PATH}, el \textit{path absoluto} de la herramienta:
      \begin{itemize}
  \item \textit{<unmkbootimg-directory>/unmkbootimg}.
      \end{itemize}
 \end{itemize}
 
 \item Una \textit{custom ROM} de Android. Descargue alguna desde \href{http://en.miui.com/}{esté}\footnote{http://en.miui.com/} link:
 \begin{itemize}
      \item Descomprima el archivo.
 \end{itemize} 
\end{itemize}


\subsection{Puesta a punto}
\begin{itemize}
    \item Desde el IDE, cree un dispositivo virtual desde el. También podrá crearlo mediante los siguientes comandos (si aparece, ingrese la letra \textit{n} cuando se le realice la pregunta ``Do you wish to create a custom hardware profile [no]''):
    \begin{lstlisting}
# avdmanager create avd --target 1 --name emulador-so
    \end{lstlisting}
    \textbf{Nota}: podrá ver la lista de \textit{targets} ejecutando:
    \begin{lstlisting}
# android list target
    \end{lstlisting}
    
    \item Inicie el emulador (\textit{avd - android virtual device}):
    \begin{lstlisting}
# emulator -avd emulador-so
    \end{lstlisting}

    \item Acceda a la shell del dispositivo:
    \begin{lstlisting}
# adb shell
    \end{lstlisting}
    Esto creara un cliente \textit{adb} e iniciara el servidor \textit{adb}, el cual se comunicará con el demonio \textit{adbd} del dispositivo ejecutandose en \textit{background}.
\end{itemize}

\subsection{SQLite}
Para su ayuda, ejecute:
\begin{lstlisting}
          # sqlite3
          sqlite> .help
\end{lstlisting}

\begin{itemize}
    \item Acceda a la configuración del dispositivo \footnote{Requiere Android Lollipop(5.0/5.1)}:
    \begin{lstlisting}
# sqlite3 /data/data/com.android.providers.settings/databases/settings.db
    \end{lstlisting}

    \item Liste todos los valores de la tabla \emph{system}:
    \begin{lstlisting}
sqlite> select * from system;
    \end{lstlisting}
    Se puede apreciar el volumen del tono de llamada, el de las notificaciones, etc.    
    
    \item Responda:
    \begin{questions}
  \question ¿Qué pasa si abrimos la configuración del dispositivo desde la interface gráfica (\emph{Menu -> Settings -> Sound}) y modificamos el volumen del tono de llamada o de notificaciones?
  
  \question ¿Y qué pasa si ejecutamos el comando \textit{delete from system;}?, ¿Sigue funcionando el sistema operativo?, ¿Qué pasa con las configuraciones?
    \end{questions}
\end{itemize}

\subsection{Almacenamiento}
\begin{itemize}
    \item Acceda, mediante \textit{SQLite}, a la información persistida por el browser. Analice la misma y borre los \textit{bookmarks}.

    \item Identifique la página de inicio del browser (\textbf{Tip:} se encuentra almacenada a través de las \textit{shared preferences}).    
\end{itemize}

\subsection{Tipo de memoria y File system}
\begin{itemize}
    \item ¿Qué tipo de memoria tiene el dispositivo virtual creado? (\textbf{Tip}: utilice el comando \textit{mount} o liste el directorio \texttt{/dev/block/})
    
    \item Agreguele almacenamiento \textit{SD} (\emph{Edit $\rightarrow$ SD Card $\rightarrow$ Size}) al dispositivo:
    \begin{lstlisting}
# android avd
    \end{lstlisting}
    ¿Aprecia algún otro tipo de memoria? ¿Por qué?   
    
    \item ¿Qué file systems existen en el dispositivo virtual creado? (\textbf{Tip}: utilice el comando \textit{mount})      
\end{itemize}

\subsection{Aplicaciones}
En Android, la forma de ejecutar aplicaciones es mediante la herramienta \textit{am}. La misma provee la funcionalidad necesaria para ejecutar activities, services, entre otras cosas.
Para su ayuda, ejecute:
\begin{lstlisting}
          # am
\end{lstlisting}

\begin{itemize}
    \item Ejecute la actividad \textit{main} de la aplicación \emph{com.android.browser}:
    \begin{lstlisting}
# am start -a android.intent.action.MAIN 
  -n com.android.browser/.BrowserActivity
    \end{lstlisting}
    \item ¿Cómo funciona?:
    \begin{itemize}
  \item El primer parámetro inica que se va a hacer. En este caso, se ejecuta una aplicación (\emph{start}).
  
  \item El segundo parámetro inica el tipo de acción que se ejecutará (\emph{android.intent.action.MAIN}). \textbf{Nota}: para ver más tipos de acciones visite \href{http://developer.android.com/guide/topics/intents/intents-filters.html}{está}\footnote{http://developer.android.com/guide/topics/intents/intents-filters.html} página.
  
  \item El tercer parámetro es la actividad que se desea mostrar 
  (\emph{com.android.browser/.BrowserActivity}).
    \end{itemize}

    \item Inicie la aplicación \emph{settings} del dispositivo:
    \begin{lstlisting}
# am start -a android.intent.action.MAIN 
  -n com.android.settings/.Settings
    \end{lstlisting}
\end{itemize}

\subsection{Procesos y Usuarios}
\begin{itemize}
    \item Abra, desde la interface gráfica, el browser que viene instalado por defecto. ¿Cómo identificamos al proceso? ¿Podemos matarlo? ¿Cómo?
    
    \item Ejecute, desde la interface gráfica, mas de una instancia del browser que viene instalado por defecto. ¿Qué puede deducir al respecto? (\textbf{Tip:} utilice el comando \textit{ps})
    
    \item Entendiendo el \textit{sandbox}:
    \begin{enumerate}
    \item Cree tres proyectos:
        \begin{itemize}
         \item nombre/dominio: unlp.so.android.uno
         \item nombre/dominio: unlp.so.android.dos
         \item nombre/dominio: unlp.so.android.tres
        \end{itemize}
    
    A partir de las verciones más recientes del SDK de Android, los proyectos deben crearse desde el mismo IDE. Si instala un versión más antigua, podrá crear los proyectos mediante los siguientes comandos \footnote{\href{http://tools.android.com/tech-docs/new-build-system/version-compatibility}{Aquí} puede ver la compatibilidad entre la versión de \textit{Gradle} y su versión del \textit{plug-in} para \textit{Android}.}:
    \begin{lstlisting}
# android create project --name soUno --path ./soUno 
--activity ActividadUno --package unlp.so.android.uno 
--target 1 -g -v 1.1.3

# android create project --name soDos --path ./soDos 
--activity ActividadDos --package unlp.so.android.dos 
--target 1 -g -v 1.1.3

# android create project --name soTres --path ./soTres 
--activity ActividadTres --package unlp.so.android.tres 
--target 1 -g -v 1.1.3
    \end{lstlisting}
    
    \item Si existe, elimine el bloque de ámbito productivo (release), del archivo \textit{build.gradle} dentro del directorio raíz de cada uno de los proyectos y/o módulos de los mismos:
    \begin{lstlisting}
buildTypes {
  release {
      runProguard false
      proguardFile getDefaultProguardFile('proguard-android.txt')
  }
}
    \end{lstlisting}
    
    \item Agregue, al \textit{tag manifest} del archivo \textit{AndroidManifest.xml} 
    
    (\emph{<directorio-raíz-proyecto>/app/src/main/AndroidManifest.xml}), el atributo \textit{sharedUserId} con el mismo valor a las aplicaciones uno y dos:
    \begin{lstlisting}
android:sharedUserId="unlp.so.android.uno"
    \end{lstlisting}
    
    \item Adicionalmente agregue, al archivo \textit{AndroidManifest.xml},  permisos de \emph{Internet} para la aplicación uno:
    \begin{lstlisting}
<manifest xmlns:android="http://schemas.android.com/apk/res/android"
    package="unlp.so.android.uno" >
    ...
    <uses-permission android:name="android.permission.INTERNET" />
    ...
</manifest>
    \end{lstlisting}
    
    \item Construya las aplicaciones mediante el IDE o bien mediante la herramienta \textit{Gradle}, posicionandose en el directorio raíz de cada proyecto (requiere conexión a Internet):
    \begin{lstlisting}
# gradle assembleDebug
    \end{lstlisting}
    Podrá ver los \textit{.apk} generados en \emph(<directorio-raíz-proyecto>/app/build/outputs/apk/). 
    
    Además podrá saber que tareas ya vienen configuradas a través de \textit{Gradle} ejecutando lo siguiente:
    \begin{lstlisting}
# gradle tasks
    \end{lstlisting}
     
    \item Instale las aplicaciones en el dispositivo (requiere que el dispositivo/emulador este conectado/iniciado):
    \begin{lstlisting}
# adb install <debug-apk-generado-proyecto-uno>
# adb install <debug-apk-generado-proyecto-dos>
# adb install <debug-apk-generado-proyecto-tres>
    \end{lstlisting}
    
    \item Ejecute cada una de las aplicaciones.
    
    \item Acceda al dispositivo mediante el \textit{adb} y responda:
    \begin{questions}
        \question ¿Qué se información puede extraer como resultado de ejecutar los siguientes comandos?:
        \begin{lstlisting}
$ adb shell
# ps        
# dumpsys package unlp.so.android.uno
# dumpsys package unlp.so.android.dos
# dumpsys package unlp.so.android.tres
        \end{lstlisting}
    \end{questions}
    
    \item Como posiblemente ya sepa, el \textit{userId} del usuario \textit{root} en Linux es el \emph{0}. En la explicación se dijo que cada aplicación Android es un usuario Linux, es decir que tiene un \textit{userId} único. Debera acceder a un archivo en el cual, entre otras cosas se define la configuración de usuarios y grupos para el sistema operativo Android. Acceda a la siguiente \href{https://android.googlesource.com/platform/system/core.git/+/master/libcutils/include/private/android\_filesystem\_config.h}{página}\footnote{https://android.googlesource.com/platform/system/core.git/+/master/libcutils/include/private/android\_filesystem\_config.h} y responda:
    \begin{questions}
        \question ¿Cuál es el \textit{userId} del usuario \emph{system}?
        
        \question ¿A partir de qué \textit{userId} Android concede indentificación para las aplicaciones de usuario?
        
        \question ¿Qué \textit{userId} se le asigna al demonio adb (\textit{adbd})?
    \end{questions}
    
    \textbf{Nota}: El que desee hacerlo puede leer \href{http://users.encs.concordia.ca/~clark/papers/2012\_spsm.pdf}{esté}\footnote{http://users.encs.concordia.ca/~clark/papers/2012\_spsm.pdf} artículo en donde se explica intrínsecamente la asignación del \textit{userId} al momento de instalar una aplicación en Android.
    \end{enumerate}
\end{itemize}

\subsection{Rooting}
\begin{questions}
  \question Siguiendo la explicación práctica y \href{http://android-dls.com/wiki/index.php?title=HOWTO%3a\_Unpack,\_Edit,\_and\_Re-Pack\_Boot\_Images}{este}\footnote{http://android-dls.com/wiki/index.php?title=HOWTO\%3a\_Unpack,\_Edit,\_and\_Re-Pack\_Boot\_Images} o \href{http://whiteboard.ping.se/Android/Rooting}{este}\footnote{http://whiteboard.ping.se/Android/Rooting} link realice el rooteo de \href{}{esta} imagen.
\end{questions}