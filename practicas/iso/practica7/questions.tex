\paragraph{Lectura recomendada: \url{https://profs.info.uaic.ro/~busaco/teach/courses/net/docs/pthreads.pdf}}

\begin{questions}

\question Describir qué es un proceso e indicar sus propiedades.
\question Describir qué es un thread e indicar sus propiedades. ¿A qué hacen
          referencia los conceptos \textit{User Level Thread} y \textit{Kernel Level Thread}?
          Indicar características principales de cada uno, así como sus diferencias.
\question Indicar al menos 5 diferencias entre los procesos y los hilos.
\question Analizar el siguiente código:
  \begin{lstlisting}
#include <pthread.h>
#include <stdio.h>

void print_message_function( void *ptr );

int main() {
  printf("Inicia el main \n");
  pthread_t thread1, thread2;
  char *message1 = "Hola";
  char *message2 = "Mundo";
  pthread_create(&thread1, NULL, (void*)&print_message_function, message1);
  pthread_create(&thread2, NULL, (void*)&print_message_function, message2);
  exit 0;
}

void print_message_function( void *ptr ) {
  char *message;
  message = (char *) ptr;
  printf("%s \n", message);
}
  \end{lstlisting}

\begin{parts}
  \part Identificar el proceso y el/los hilo/s.
  \part Compilar el programa utilizando el siguiente comando:
    \begin{lstlisting}
      $ gcc HelloWorld.c -o HelloWorld  -lpthread
    \end{lstlisting}
    \textit{Aclaración: la opción -l especifica que nuestro código utiliza una
            librería, en este caso \textbf{pthreads}. En caso de que \textbf{pthreads}
            no se encuentre instalada, debemos instalarla. En el caso de Debian:}
            \begin{lstlisting}
              # apt-get install libpthread-stubs0-dev
            \end{lstlisting}
  \part Ejecutar el programa reiteradas veces (al menos 15 ó 20):
    \begin{lstlisting}
      $ ./HelloWorld
    \end{lstlisting}
    ¿Cuál es el resultado que obtiene? Indicar al menos 2 inconvenientes que
    pueda generar la ejecución del código compilado.
\end{parts}

\question Modificar el código del inciso anterior con el fin de planificar la ejecución de hilos. Para ello utilice la sentencia \textit{sleep}. ¿Qué inconvenientes encuentra? Explicar las razones y analizar posibles alternativas para solucionar el inconveniente planteado.

\question Modificar nuevamente el código del inciso 4 de modo tal que la función \textit{main} registre e imprima en pantalla cuál fue el primer thread en ejecutarse. ¿Cómo lo haría? ¿Encuentra algún problema en su solución?

\end{questions}
