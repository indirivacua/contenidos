\section{Overview}

\subsection{Principales características}
\begin{frame}
  \frametitle{Principales características}
  \begin{itemize}
    \item Android corre sobre Linux (v4.4.1).
    
    \item Android se aprovecha de Linux para:
      \begin{itemize}
  \item Abstracción de hardware.
  \item Administración de memoria.
  \item Administración de CPU.
  \item Networking.
  \item Seguridad.
      \end{itemize}
    
    \item El usuario no nota el Linux subyacente.
  \end{itemize}
\end{frame}

\subsection{Etimología e Historia}
\begin{frame}
  \frametitle{Eimología e Historia}
  \begin{itemize}
    \item ¿Sueñan los androides con ovejas eléctricas? - Philip K. Dick
    \begin{figure}
  \includegraphics[scale=0.1]{images/andy.png}
  \caption{Andy}
    \end{figure}
    
    \item Android Inc. $\rightarrow$ 2003.
    
    \item Google $\rightarrow$ 2005.
    
    \item OHA (Open Handset Alliance) $\rightarrow$ 2007.
    
    \item Versión 1.0 (\emph{Apple Pie}) estable $\rightarrow$ 2008.
  \end{itemize}
\end{frame}

\subsection{Versionado}
\begin{frame}
  \frametitle{Versionado \href{https://es.wikipedia.org/wiki/Anexo:Historial\_de\_versiones\_de\_Android}{[ref]}}
  \begin{itemize}
    \item Diversas versiones simbólicas y númericas.
    
    \item Nombre de dulces.
    
    \item v4.4 (KitKat), v5.1 (Lollipop), v6.0 (Marshmallow), v7.0 (Nougat), v8.0 (Oreo), v9.0 (P próximamente).
  \end{itemize}
\end{frame}

\subsection{Stack}
\begin{frame}
  \frametitle{Stack}
  \begin{figure}
      \includegraphics[scale=0.1]{images/stack2.png}
  \end{figure}
\end{frame}

\subsection{Procesos e Hilos}
\begin{frame}
  \frametitle{Procesos e Hilos - Componentes \href{https://developer.android.com/guide/components/processes-and-threads}{[ref]}}
  \begin{itemize}
    \item Componentes: \textit{activity}, \textit{service}, \textit{rceiver}, \textit{provider}.

    \item Cuando se invoca el primer componente de una aplicación se crea un proceso Linux con un único thread (\textit{main}/\textit{UI} thread).
    
    \item Por defecto todos los componentes de una aplicación se ejecutan sobre ese mismo thread del proceso que represente a la aplicación.
    
    \item Los componentes pueden ser configurados para que ejecuten sobre diferentes procesos (\textit{android:process}).
    
    \item Es posible crear threads dentro de cada proceso $\rightarrow$ importante para el acceso a la \textit{UI} (\emph{UI toolkit thread unsafe}).
  \end{itemize}
\end{frame}

\begin{frame}
  \frametitle{Procesos e Hilos - Clasificación}
  \begin{itemize}
    \item Android elimina procesos bajo demanda ante la ausencia de memoria $\rightarrow$ no cuenta con área de intercambio \footnote{https://zerocredibility.wordpress.com/2009/08/24/why-android-swap-doesnt-make-sense/}.
    
    \item Decición en base a jerarquía de procesos clasificados en:
    \begin{itemize}
     \item \textit{Foreground process}
     \item \textit{Visible process} $\rightarrow$ \textit{onPause()}
     \item \textit{Service process} $\rightarrow$ \textit{startService()}
     \item \textit{Background process} (\emph{LRU}) $\rightarrow$ \textit{onStop()}
     \item \textit{Empty process}
    \end{itemize}       
  \end{itemize}
\end{frame}

\begin{frame}
  \frametitle{Procesos e Hilos - Seudo swapping}
  \begin{figure}
      \centering
      \includegraphics[scale=0.5]{images/restore-instance.png}
  \end{figure}
\end{frame}

\begin{frame}
  \frametitle{Procesos e Hilos - DVM/ART}
  \begin{itemize}
      \item Individualizada por cada aplicación.
      
      \item Cada proceso es un proceso Linux.
      
      \item Cada thread es un thread Linux.
      
      \begin{figure}
  \centering
  \includegraphics[scale=0.3]{images/dvm-linux-processes.png}
      \end{figure}
  \end{itemize}
\end{frame}

\subsection{DVM/ART vs. JVM}
\begin{frame}
  \frametitle{DVM/ART vs. JVM}  
  \begin{table}
      \centering
      \resizebox{\textwidth}{!}{
    \begin{tabular}{| c | c |}
        \hline
        \bf DVM/ART & \bf JVM \\
        \hline
        Máquina basada en registros & Máquina basada en stack \\
        \hline
        Diseñada para ejecutar sobre poca memoria & Consume más memoria \\
        \hline
        Ejecuta sus propios bytes codes (\textit{.dex}) \footnote{https://source.android.com/devices/tech/dalvik/dex-format} & Ejecuta bytes codes Java (\textit{.class}) \\
        \hline 
        Uni-plataforma $\rightarrow$ Android & Multi-plataforma \\
        \hline
        Ejecutable $\rightarrow$ \textit{.apk} & Ejecutable $\rightarrow$ \textit{.jar} \\
        \hline
    \end{tabular}
      }
  \end{table}

  Mejoras de \textit{ART} \href{https://source.android.com/devices/tech/dalvik/}{[ref]}:
  \begin{itemize}
      \item Incorporación de compilación \textit{Ahead-of-time} (AOT) utilizando \textbf{dex2oat} además de la compilación \textit{just-in-time} (JIT) existente.
      
      \item Optimización del garbage collection (GC).
  \end{itemize}

\end{frame}

\subsection{Aplicaciones}
\begin{frame}
  \frametitle{Aplicaciones - Building \href{https://developer.android.com/studio/build/}{[ref]}}
  \begin{itemize}
    \item Un \textit{android package} contiene todo lo necesario para ejecutar la aplicación en un dispositivo.
    
    \item \textbf{Gradle} es el \textit{application builder} oficial del \textit{SDK} de Android (\textit{build.gradle}).
    \begin{itemize}
  \item Debug mode $\rightarrow$ automático.
  \item Release mode.
    \end{itemize}
  \end{itemize}
  
  \begin{figure}
    \centering
    \includegraphics[scale=0.4]{images/build-app.png}
  \end{figure} 
\end{frame}

\begin{frame}
  \frametitle{Aplicaciones - Installing}
  \begin{figure}
    \centering
    \includegraphics[scale=0.3]{images/install-app.png}
  \end{figure} 
\end{frame}

\begin{frame}
  \frametitle{Aplicaciones - Launching}
    \begin{figure}
      \centering
      \includegraphics[scale=0.5]{images/launch-app.jpg}
    \end{figure}
\end{frame}

\subsection{Seguridad}
\begin{frame}
  \frametitle{Seguridad - Datos de aplicación}
  \begin{itemize}
    \item Ninguna aplicación puede ejecutar operaciones que afecten a las demás.
    
    \item Solo pueden escribir y leer datos privados de la aplicación.
    
    \item Las aplicaciones comparten datos de manera explícita.    
  \end{itemize}
\end{frame}

\begin{frame}[fragile]
  \frametitle{Seguridad - Id. de usuario y permisos sobre archivos}
  \begin{itemize}
    \item Cuando se instala un \textit{.apk}, Android le otorga un \textit{userId} de Linux definitivo.
    
    \item En otro dispositivo el mismo paquete podría tener otro \textit{userId}.
    
    \item Dos aplicaciones con el mismo \textit{userId} son tratadas como la misma aplicación.
    
    \item Mismo \textit{userId} = Mismo usuario Linux = Misma aplicación = Mismos permisos (\textit{UGO}) sobre los archivos de la aplicación.
    
    \item Las aplicaciones que comparten el \textit{userId} tienen que compartir la firma. Es decir que deben ser firmados por la misma clave privada.

    \begin{lstlisting}
# ls -l data/data/ | grep brow
# drwxr-x--x u0_a14 u0_a14 2016-05-01 17:55 com.android.browse
    \end{lstlisting}
  \end{itemize}
\end{frame}

\begin{frame}
  \frametitle{Seguridad - Ejemplo shared user id}
  \begin{figure}
    \centering
    \includegraphics[scale=0.4]{images/two-same-apps.jpg}
  \end{figure}
\end{frame}

\begin{frame}[fragile]
  \frametitle{Seguridad - Recursos}
  \begin{itemize}
    \item Se debe declarar el acceso a los recursos de manera estática (\textit{Manifest.xml}) $\rightarrow$ dinámico a partir de v6.0
    
    \item Cuando la aplicación es instalada el usuario debe dar su consentimiento.
    
    \item \textit{SecurityException}
    \begin{lstlisting}
<manifest xmlns:android="http://schemas.android.com/apk/res/android"
    package="com.android.app.myapp" >

    <uses-permission android:name="android.permission.RECEIVE_SMS" />
    <uses-permission android:name="android.permission.INTERNET" />
    ...
</manifest>
    \end{lstlisting}    
  \end{itemize}
\end{frame}

\begin{frame}
  \frametitle{Seguridad - Application sandbox}
  \begin{figure}
    \centering
    \includegraphics[scale=0.2]{images/sandbox.png}
  \end{figure}
  
  \begin{itemize}
      \item Solo una porción de código Android corre como \textit{root}.
      
      \item ¿Cómo hacen diferentes reproductores para acceder a la música?
  \end{itemize}
\end{frame}

\begin{frame}
  \frametitle{Seguridad - Signing \href{https://developer.android.com/studio/publish/app-signing}{[ref]}}
  \begin{itemize}
    \item Todas las aplicaciones (\textit{.apk}) tienen que ser firmados digitalmente.

    \item Se emiten certificados auto-firmados (\textit{KeyTool}):
    \begin{itemize}
  \item Debug mode $\rightarrow$ \textit{debug key} $\rightarrow$ el \textit{keystore} se crea automáticamente (\textit{\$HOME/.android/debug.keystore}).
  
  \item Release mode $\rightarrow$ \textit{developer's private key} $\rightarrow$ manualmente.
    \end{itemize}    
  \end{itemize}
\end{frame}

\begin{frame}
  \frametitle{Seguridad - Repaso y Overview}
  \begin{itemize}
   \item A nivel de \textit{AndroidManifest} $\rightarrow$ permisos a recursos que termina de conceder o configurar el usuario (aplicado por el \textit{sandbox}).
   
   \item A nivel de las aplicaciones y sus archivos $\rightarrow$ certificado, \textit{userId} y permisos \textit{UGO} (aplicado por el \textit{sandbox}).
   
   \item A nivel de file system $\rightarrow$ particiones \textit{read-only} (/system).
   
   \item A nivel de \textit{Linux} y \textit{ART} (aislamiento de aplicaciones).
  \end{itemize}
\end{frame}

\subsection{Almacenamiento}
\begin{frame}
  \frametitle{Almacenamiento \href{https://developer.android.com/guide/topics/data/data-storage}{[ref]}}
  \begin{itemize}
      \item Almacenamiento interno
      \begin{itemize}    
          \item \textit{MODE\_PRIVATE}, \textit{MODE\_WORLD\_READABLE} o \textit{MODE\_WORLD\_WRITEABLE} $\rightarrow$ desde v7.0 \textit{FileProvider}.
      \end{itemize}

      \item Shared preferences \href{https://developer.android.com/guide/topics/data/data-storage.html}{[ref]}
      \begin{itemize}
    \item Se puede manejar una colección de ellas.
    
    \item Se almacenan a través de archivos XML.
        
    \item Aplican como almacenamiento interno.
    
    \item \textit{/data/data/$<$package\_name$>$/shared\_prefs/}
      \end{itemize}
    \item Almacenamiento externo $\rightarrow$ READ\_EXTERNAL\_STORAGE y WRITE\_EXTERNAL\_STORAGE
      \item SQLite: 
      \begin{itemize}
    \item Bajo licencia \textit{GPL}.
    
    \item Actúa sobre archivos ordinarios.

    \item Cumple las propiedades ACID.
    
    \item \textit{/data/data/$<$package\_name$>$/databases/}
      \end{itemize}
  \end{itemize}
\end{frame}

\subsection{File System}
\begin{frame}[fragile]
  \frametitle{File system - Tipos de memorias}
  \begin{itemize}
      \item Raw NAND flash:
      \begin{itemize}
    \item Subsistema \textit{MTD} (Memory Tecnology Device).
    
    \item \textit{/dev/block/mtdblockN}
    
    \begin{lstlisting}
$ cat /proc/mtd                                                 
dev:    size   erasesize  name
mtd0: 05660000 00020000 "system"
mtd1: 04000000 00020000 "userdata"
mtd2: 04000000 00020000 "cache"
    \end{lstlisting}
      \end{itemize} 
      
      \item SD, MicroSD y eMMC (Flash Translation Layer):
      \begin{itemize}
    \item Driver \textit{mmcblock} (Multimedia Card Block).
    
    \item \textit{/dev/block/mmcblk[chip number]p[partition number]}
    
    \begin{lstlisting}
$ ls -l /dev/block/platform/msm_sdcc.1/by-name
cache -> /dev/block/mmcblk0p36
system -> /dev/block/mmcblk0p38
userdata -> /dev/block/mmcblk0p40
    \end{lstlisting}
      \end{itemize}
  \end{itemize}
\end{frame}

\begin{frame}
  \frametitle{File system - Tipos}
  \begin{itemize}
      \item YAFFS
      \begin{itemize}
  \item Booteo más rápido.
  
  \item Consume menos memoria.
  
  \item Divide los archivos en páginas.
  
  \item \textit{GPL}.
  
  \item v1 y v2.
  
  \item Single-theading.
  
  \item Por lo general, hasta \textbf{Froyo} (v2.2).
  
  \item Especializado para funcionar en memorias de tipo raw NAND.
      \end{itemize}
      
      \item Ext4
      \begin{itemize}
  \item Multi-theading.
  
  \item Desde \textbf{Gingerbread} (v2.3).
  
  \item Utilizado generalmente en memorias SD, MicroSD o eMMC.
      \end{itemize}   
  \end{itemize}
\end{frame}

\begin{frame}[fragile]
  \frametitle{File system - Puntos de montaje principales}
  \begin{itemize}
      \item \textit{/system}
      
      \item \textit{/system/bin}
      
      \item \textit{/data/app}
      
      \item \textit{/data/data/$<$package\_identifier$>$}
      
      \item \textit{/recovery}
      
      \item \textit{/boot}
      
      \item \textit{/cache}
  \end{itemize}
\end{frame}

\subsection{Licencia}
\begin{frame}
  \frametitle{Licencia}
  \begin{itemize}
      \item La idea de \textbf{Google} es mantener \textit{GPL} fuera del espacio de usuario:
      \begin{itemize}
    \item \textit{Apache v2} para el espacio de usuario.
    
    \item \textit{GPL} para el espacio de kernel.
      \end{itemize}
      
      \item Además de tamaño y optimización, esta fue otra de las razones por las que implementaron su propia versión de \textit{libc} (\textit{BIONIC}).
  \end{itemize}
\end{frame}